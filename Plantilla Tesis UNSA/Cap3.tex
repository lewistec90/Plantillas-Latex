\chapter{Formalismos y/o teor�a propuesta}
\hrule \bigskip \vspace*{1cm}
%------------------------------------------------------------------------

Seria recomendable que cada uno de las etapas o puntos principales
vayan acompa�ados de una discusi�n (mini conclusi�n) detallando y
justificando la raz�n de su existencia.

\section{Instalaci�n de \LaTeX}

Debemos iniciar la instalaci�n mediante los siguientes paquetes
b�sicos, es recomendado seguir el siguiente orden en la
instalaci�n:

\begin{description}
    \item[AFPLGhostscript] Nos permite trabajar con los formatos
    EPS que caracterizan a \LaTeX (Free).
    \item[GSview] Para visualizar los PS y EPS
    \item[Acrobat Reader] Para visualizar los PDF (Free).
    \item[small-miktex] el compilador y los \verb"packages" del
    \LaTeX (Free).
    \item[WinEdt] Un potente editor para \LaTeX.
\end{description}

Estos paquetes son opcionales, pero muy �tiles:

\begin{description}
        \item[Diccionario] Diccionario para poder corregir en
    espa�ol, a�n incompleto solo en WinEdt (Free).
     \item[GNUplot]Poderoso Graficador y procesador matem�tico, muy usado
     en los trabajos de investigaci�n y tesis(Free).
\end{description}
