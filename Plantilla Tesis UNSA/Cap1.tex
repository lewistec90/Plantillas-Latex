\chapter{Introducci�n}
\hrule \bigskip \vspace*{1cm}
%------------------------------------------------------------------------


Este es el ejemplo de un cap�tulo de tu tesis. Aqu� deber�as
escribir la introducci�n de tu tesis. Junto con este archivo
deber�s encontrar otros tutoriales con m�s informaci�n de c�mo
utilizar los diferentes paquetes y su sintaxis.

Esta es una ``posible'' estructura de tesis recomendada. Los t�tulos
de los cap�tulos solo son referentes a lo que deben tener el
contenido, pero no es necesariamente la m�s apropiada a tu tesis,
los expertos recomiendan no sobrecargar mucho la tesis con
informaci�n externa, es decir el conocido ``Marco te�rico'' en el
cap�tulo 2 ya que esta puede ser ampliamente referenciado en la
bibliograf�a.

\section{Contexto y Motivaci�n}

�cual es el �mbito en que esto se desenvuelve? y �que necesidad
existe para motivar una investigaci�n en tu tema?

\section{Definici�n del problema}

As� como ya esta, �que problema existe actualmente?, respecto a lo
que quieres proponer o mostrar.

\section{Justificaci�n}

�porque estas desarrollando esta tesis?

\section{Objetivos}

�Que pretendes obtener o resolver? y en los objetivos espec�ficos
detallar cada una de las tareas que realizaras, en este punto debes
ser muy exacto y concreto(recuerda que en base a esto justificaras
si estas obteniendo resultados)

\section{Organizaci�n de la tesis}

Una breve descripci�n de cada uno de los cap�tulos que estas
desarrollando desde el CAP 2 hasta el capitulo antes del ap�ndice.
